

\chapter{测试框架设计}

在本部分中,将对整个测试框架的结构进行设计以及该设计相对于已有的一些测试框架所具有的优势。

\section{总体设计}


\subsection{内核编译系统}
内核的编译是一个非常耗时的过程,要想进行高效率的性能测试工作,那么高效率的内核编译是必不可少的。

\subsection{性能测试配置}
要进行性能测试,自然是需要有能够用于运行待测试内核的测试机,为了能够尽可能地提高性能测试效率,在我们的测试框架中,我们将使用具有Linux内核支持的KVM虚拟机来进行相关的测试工作。

在一台真实的计算机上,可以同时运行多台使用KVM技术的虚拟机,更为方便的是KVM虚拟机支持使用指定的内核和内存镜像文件来运行,这样,我们就可以不用花费大量的时间来进行内核镜像和内存镜像文件的打包,因此,使用虚拟机来进行内核的性能测试能够大大提高测试的效率。


\subsection{测试运行流程}

在多个测试服务器之间相互通信。

\subsection{问题自动定位}
在我们的测试框架中,我们将会对某个版本的内核进行测试,并使用一定的算法来处理这些测试数据,从中找到性能回退问题。

一旦找到性能回退的问题,我们会再次进行测试以确认这个回退问题,在确认性能回退问题之后,我们将会开始对造成这个性能回退问题的代码进行定位。

一般来说,性能回退问题往往是由某一次或者多次的修改造成的,我们的目标就是定位到造成性能回退问题的这一次修改。



\section{提升与优势}

我们新设计的框架相对于已有的一些测试框架和具有相似功能的框架相比具有很大的提升和更多的优势。

\begin{itemize}
\item 与LKP相比

\begin{enumerate}
\item 覆盖更多的开发分支,能够更早地发现潜在的性能回退
\item 具有更高的测试效率
\item 拥有更加全面和完善的数据处理和分析能力
\item 提供更加详细,更加丰富的测试结果报告
\item 尽可能充分利用已有的计算资源
\end{enumerate}

\item 与MMTests相比

\begin{enumerate}
\item 拥有更为完善和统一的数据处理流程
\item 拥有更全面的测试数据覆盖
\item 具有更完整的测试流程
\item 具有数据分析功能
\item 具有问题定位功能
\end{enumerate}

\end{itemize}