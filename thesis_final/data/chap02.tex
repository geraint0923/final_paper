
\chapter{相关工作}

\section{相关的理论研究工作}

\subsection{利用社区提高开源软件质量}

\subsection{自动性能分析}

\section{相关工程项目}

Linux内核作为一个庞大和复杂的系统软件,很早之前就已经引起了很多公司或团体的注意,并以之为测试目标,开发了比较实用,比较方便的进行Linux性能测试的工程项目,比如LKP\cite{chen2007keeping}和MMTests就是其中两个比较优秀的项目。

\subsection{Linux Kernel Performance}

Linux Kernel Performance(简称LKP),是Intel公司的开源技术中心于2005年开始的一个用于进行Linux内核性能测试的一个测试框架。

由于Linux内核的快速发展,Intel开源技术中心的一部分工程师开发了一套对Linux内核性能进行测试的框架,这套框架的具体架构如图\ref{fig:old_lkp_arch}

\begin{figure}[H]
\centering
\includegraphics[width=10cm]{old_lkp_arch}
\caption{LKP测试框架架构}
\label{fig:old_lkp_arch}
\end{figure}

在这套框架中,主要的测试分为5个阶段,分别是:
\begin{enumerate}
\item 下载代码并编译。首先从linux.org网站上面下载各个版本的内核源代码,并进行编译,一般都只是对具有tag的版本(包括RC版本和最终发行版)进行编译。
\item 运行测试。使用上一个阶段中编译好的内核来启动一台物理机器,并在这一台物理机器中运行各种各样的测试例程,如Kbuild,Reaim7,Netperf,Tbench等比较经典和常用的测试例程。同时,在测试例程运行的过程中,还会启动vmstat,iostat,sar,ps等工具来进行测试数据的采样和收集
\item 结果分析。根据在上一个阶段中收集到的测试数据,作出相关的数据统计分析,同时结合每一个测试例程的运行结果来给每一次的测试进行评分。在评分的过程中,给予每一个不同的测试例程相应的权重值,然后根据权重值计算出本次测试的得分,然后通过得分来判定性能的变化情况。
\item 问题查找。如果在进行结果分析的时候发现了性能回退的情况,那么就会在Linux内核的代码树中使用git-bisect来进行问题的查找,定位出问题发生的位置和原因。
\item 结果呈现。
\end{enumerate}

\subsection{MMTests}