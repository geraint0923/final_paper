
\chapter{引言}


\section{研究背景}

在这个信息技术行业高速发展的年代,软件的开发方式逐渐发生改变,软件的开发速度也越来越快,而且软件的代码量也逐渐增多,同时社会对软件质量的要求也是越来越高,这样也就相应地提高了对软件开发管理的要求,其中对软件性能的要求更为苛刻,而操作系统作为一种使用率极高的系统软件,对性能的控制和管理更是追求极致,特别是像Linux这样贡献者众多开源操作系统,因此,设计并开发一个能够比较快速地进行Linux性能测试并且从中定位出问题的测试系统是很有意义的并且也是非常有必要的。

本文中针对Linux内核的发展趋势和开发模式,设计了一套对Linux内核性能进行高效测试的测试框架(系统)。在本文设计的这一套测试框架中,可以进行多方面的性能测试,并在测试的过程中以尽可能低的系统性能消耗来采集大量的系统性能数据,然后对采集到的数据进行合理的分析和判断,并利用Linux内核代码的版本管理工具git来进行不同版本代码的测试,从而在出现性能问题之后帮助开发者定位问题的所在,大大减轻Linux内核开发者的开发和测试负担,提高Linux内核的开发效率,并有效降低Linux内核出现性能问题的可能性。


\begin{figure}[H]
\centering
\includegraphics[width=10cm]{pactches_count}
\caption{每年提交的Linux补丁数量}
\label{fig:pactches_count}
\end{figure}


\subsection{Linux的开发模式}

\begin{figure}[H]
\centering
\includegraphics[width=10cm]{linux_kernel_change_flow}
\caption{Linux内核开发模式}
\label{fig:linux_kernel_change_flow}
\end{figure}

\subsection{Linux的开发特点}

随着Linux内核开发的全球化,Linux开发的步伐也逐渐加快,开发的规模也逐渐扩大,目前内核开发大致具有以下的特点:

\begin{enumerate}
\item 越来越多的新功能被添加到内核中来,导致内核的结构越来越复杂
\item 内核的复杂结构使得任何一点改动对内核性能造成影响的可能性加大
\item Linux内核进行官方测试的间隔一般比较大(一般旨在有新版本发布的时候才会进行比较完整的测试)
\item 一旦发现版内核中出现性能回退的问题,这个问题就会随着各大Linux发行版扩散到广大的用户当中
\end{enumerate}

\section{课题目标}

正如研究背景所述,目前Linux内核的开发具有比较鲜明的特点,而且Linux作为一个操作系统内核,也具有一点的特殊性,因此,很有必要设计并实现出一套与Linux内核相适应的性能测试框架,方便Linux的开发者们更加方便地进行性能回退问题的管理和处理。

本课题的目的就是设计并实现一套能够进行Linux性能测试并具有如下特征的框架(系统):

\begin{enumerate}
\item 较快并准确进行Linux性能测试
\item 在出现性能回退之后,多次运行进行问题确认
\item 确认性能回退之后,能够最快地定位到出现问题的代码
\item 在定位到问题代码之后,将相关的测试数据及出现问题的代码通知给相关代码的作者
\end{enumerate}