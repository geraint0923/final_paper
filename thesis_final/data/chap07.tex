

\chapter{总结}
在本毕业设计中,我和Intel的工程师一起设计实现了一套对Linux内核进行性能测试的框架,并在某一些细节上进行优化。所完成的工作具体如下:
\begin{enumerate}
\item 以简单和模块化强为原则,设计整套测试框架,并明确每一个框架中每一个模块的功能
\item 实现多种系统监视器,使得在测试中能够采集到的数据更加丰富和全面,以便捕捉到细微的性能回退问题
\item 完成所有系统监视器信息的提取和格式化,对测试结果进行有条理的管理
\item 实现的测试结果的测试与分析,为进行性能回退判断做好准备,并将比较的结果可视化呈现出来
\item 实现性能回退判断算法用于寻找性能回退问题,并对判断算法进行进一步的优化,使其判断结果更加可靠
\end{enumerate}

在本毕业设计中设计实现的这套Linux内核测试框架是非常具有实用价值的,并且将在全部完成之后投入使用,真正开始为Linux内核寻找内核回退问题。随着Linux内核在越来越多的计算机上部署和实用,Linux内核性能所受到的关注度也越来越高,各种各样的Linux内核性能测试系统也不断被开发出来,如LKP\cite{chen2007keeping}和TKO\cite{bligh2006fully},但是这些系统都存在这样那样的问题,效果并不理想,而在本毕业设计中设计的Linux内核性能测试框架,则是在总结前人的经验和教训的基础上,重新实现的一套测试框架,参与设计的Intel的工程师中还包括了具有丰富经验的Linux内核核心开发者,在设计的时候更多地从内核开发这的角度出发,力图将整个框架打造成一个极具实用性的性能测试框架。


Linux内核作为一个操作系统内核,其结构和代码都是极端复杂的,而且其对性能的要求也是及其严苛的,一套性能测试框架如果能够有效地对Linux内核进行性能测试,那么对于其他软件的性能测试也将是适用的。在将来,可以考虑借鉴这样的思路搭建一套适用与清华大学的一个教学型操作系统uCore,从而对uCore进行性能测试,帮助uCore提升性能。