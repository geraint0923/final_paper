

\chapter{数据的提取与处理}

要检测性能回退是否出现,我们必须要从测试的环境中获取能够对性能进行判断的相关数据,即我们要进行测试数据的提取。

\section{数据提取}

虽然在已有的研究结果\cite{Jiang:2010:AAL:1831708.1831726}中指出使用额外的监视器来获取测试数据很有可能会造成额外的消耗从而影响测试数据的可靠性,但是在我们的测试框架中,我们采用最简单的方式来实现监视器,并且使用比较低的采样频率(每秒一次)来获取测试数据,以此来降低系统监视器对测试结果造成的影响。

\subsection{系统监视器}

系统监视器,就是以某种频率从系统中读取系统状态参数的并将其记录下来的程序。在这一部分,我将详细描述我们框架中的系统监视器的设计和实现。

前面提到,在我们的测试框架中,我们将采集尽可能多的,覆盖面广的测试数据,这样,我们才进行更加细致和全面的分析,于是在我们的测试框架中,将包含表\ref{tab:monitor_type}几类的系统监视器。

\begin{table}[tbp]
\centering  % 表居中
\tablehead{\hline
	监视器类型 & 监视器功能\\
	\hline \hline}
\tabletail{\hline}

\begin{supertabular}{p{3cm}p{10cm}}
CPU类 & 用于监视和CPU相关的系统状态\\
内存类 & 主要用于监视系统中和内存管理相关得系统参数\\
I/O类 & 主要用于监视系统中得输入输出状态\\
杂类 & 监控出了CPU,内存和I/O之外得其他参数\\
\end{supertabular}
\caption{监视器分类}
\label{tab:monitor_type}
\end{table}

下面我们将分类介绍这些不同的系统监视器。
\subsubsection{CPU类系统监视器}


对于CPU类的系统监视器来说,比较重要的自然是监视系统中的CPU的负载情况和CPU负载中的不同成分(包括内核系统负载和用户态负载),在多核SMP的情况下,我们还需要分别监视每一个核的运行情况,另外,在这里,我们把进程相关的监视器也包含在CPU类系统监视器中。

表\ref{tab:cpu_monitor}中列出了在我们的框架中所使用的所有CPU类系统监视器,以及它们所能够采集的数据。

\begin{table}[tbp]
\centering  % 表居中
\tablehead{\hline
	监视器 & 监视器功能 & 相关文件\\
	\hline \hline}
\tabletail{\hline}

\begin{supertabular}{p{3cm}p{7cm}p{3cm}}
interrupts & 监视系统中每一个CPU核中各种中断发生的情况 & /proc/interrupts\\
sched\_debug & 监视系统中每一个CPU核的各种调试参数和系统信息,如时钟的频率等;此外,该监视器还监视系统中的进程运行情况,提供进程调度的相关信息 & /proc/sched\_debug\\
softirqs & 监视系统中所有CPU核上的所有软中断情况 & /proc/softirqs\\
vmstat & 使用Linux中自带的常用工具vmstat来获取当前系统中的CPU负载状态,其中将包括 & 无\\
\end{supertabular}
\caption{CPU类系统监视器}
\label{tab:cpu_monitor}
\end{table}

在操作系统中,CPU作为执行指令的工具,其功能十分复杂,中断则是CPU最重要的功能之一,了解系统的中断有助于我们了解分析系统中相关硬件和相关子系统的运行状况,在发生性能回退问题的时候能够帮助我们进行更加准确地分析和定位问题。


\subsubsection{内存类系统监视器}

内存管理是操作系统中的一个重要功能,于是内存管理子系统也成为操作系统的中最为重要的子系统之一。内存管理子系统的性能好坏也直接影响整个操作系统的性能,因此,对于一个性能测试框架来说,内存管理性能相关的参数自然是必不可少的。

在Linux内核中,内存管理子系统是相当复杂的。其中,在内存管理中,比较重要的一个内存管理算法就是buddy算法,buddy system正是使用了这个这个算法来管理内存,这个算法的性能好坏也直接影响到整个内存管理系统,因此我们使用了一些监视器来专门监视buddy算法的运行状况。

此外,Linux内核的内存管理子系统还包括了非一致内存访问(NUMA)的管理。因此,除了监视普通的内存管理系统之外,还需要对NUMA管理系统也进行监视。

为了尽可能覆盖Linux内核中内存管理子系统中的相关功能,在我们的性能测试框架中,选择了表\ref{tab:mm_monitor}中列出的系统监视器为我们提供测试数据。

\begin{table}[tbp]
\centering  % 表居中
\tablehead{\hline
	监视器 & 监视器功能 & 相关文件\\
	\hline \hline}
\tabletail{\hline}

\begin{supertabular}{p{3cm}p{7cm}p{3cm}}
buddyinfo & 监视操作系统内存管理子系统中的Buddy内存分配算法的运行状况,其中的信息既可以用来进行性能的判断计算,还可以方便内核开发者的调试和问题定位 & /proc/buddyinfo\\
meminfo & 监视操作系统中内存各方面的使用状况 & /proc/meminfo\\
slabinfo & 监视内存管理子系统中SLAB算法的运行状况,功能和buddyinfo类似,可以提供测试数据,也可方便开发者的调试 & /proc/slabinfo\\
pagetypeinfo & 和buddyinfo监视器类似,其中还收集NUMA中不同node的详细信息 & /proc/pagetypeinfo\\
zoneinfo & 在NUMA系统中收集信息,可以得到NUMA中每一个node的详细信息 & /proc/zoneinfo\\
numa-meminfo & 监视NUMA系统中每一个node的内存信息 & /sys/devices/system/node\\
numa-vmstat & 收集NUMA系统中每一个node的虚拟内存信息 & /sys/devices/system/node\\
numa-numastat & 监视NUMA中每一个node的详细运行状态 & /sys/devices/system/node\\
proc-vmstat & 提供系统中相关内存子系统的相关信息,包括内存管理和页面切换管理等 &/proc/vmstat\\
\end{supertabular}
\caption{内存类系统监视器}
\label{tab:mem_monitor}
\end{table}



\subsubsection{I/O类系统监视器}

输入输出(I/O)几乎是计算机里最为重要的一部分了,同样,对于一个操作系统来说,输入输出(I/O)子系统同样也是最为重要的一部分。

一般,I/O子系统一般包括外设的输入输出和网络操作,我们选择的系统监视器也将监视这几个方面的测试数据。

表\ref{tab:io_monitor}中列出了我们的测试框架中使用的I/O类系统监视器。

\begin{table}[tbp]
\centering  % 表居中
\tablehead{\hline
	监视器 & 监视器功能 & 相关文件\\
	\hline \hline}
\tabletail{\hline}

\begin{supertabular}{p{3cm}p{7cm}p{3cm}}
mountstats & 监视操作系统中文件系统的挂载情况,其中会给出NFS文件系统的详细运行状态,这可以对操作系统中NFS的性能进行比较详细的监视 & /proc/self/mountstats\\
nfsstat & 这是Linux系统中常用的一款NFS状态查看工具,我们将其封装成一个完整的系统监视器,可以对NFS的运行状态进行详细的汇报,有助于进行性能分析 & 无\\
iostat & 同样也是一款Linux的常用工具,其作用在于 & 无\\
\end{supertabular}
\caption{I/O类系统监视器}
\label{tab:io_monitor}
\end{table}


\subsubsection{杂类系统监视器}

除了CPU类,内存类以及I/O类三类系统监视器之外,我们还选用了一些同样非常有用的监视器,这些监视器能够帮助我们更加方便和快速地分析和定位问题,表\ref{tab:misc_monitor}列出了我们的测试框架中所使用的所有杂类系统监视器。

\begin{table}[tbp]
\centering  % 表居中
\tablehead{\hline
	监视器 & 监视器功能 & 相关文件\\
	\hline \hline}
\tabletail{\hline}

\begin{supertabular}{p{3cm}p{7cm}p{3cm}}
lock\_stat & 监视操作系统内核中各种锁的使用情况,其中包括每一个锁lock和unlock的次数,以及每一个锁的总耗时等相关信息 & /proc/lock\_stat\\
latency\_stats & 监视当前内核中函数的调用延迟信息和函数的调用栈关系 & /proc/latency\_stats\\
\end{supertabular}
\caption{杂类系统监视器}
\label{tab:misc_monitor}
\end{table}

这里我们选择了lock\_stat作为我们框架中的一个杂类系统监视器是因为在Linux操作同内核中已经实现了锁的机制,而且在内核中,锁也得到了大量的应用,在Boyd-Wickizer的论文\cite{boyd2012non}中提到,锁机制的实现对操作系统内核性能的影响非常大,特别是在多核的计算机上,锁的实现对系统内核的影响更是巨大。因此,对内核中的锁进行监视是非常有必要的,而在Linux内核中最好的监视器可以说就是lock\_stat。

在杂类系统监视器中,我们还选用了latency\_stats,latency\_stats是由Intel开发的一款Linux内核调用信息记录工具,这个监视器将会输出如下面的内容:
{
\footnotesize
\begin{verbatim}

time: 1365434000.208857599
Latency Top version : v0.1
1 1537 1537 pipe_wait pipe_read do_sync_read vfs_read sys_read system_call_fastpath
1 65 65 do_wait sys_wait4 system_call_fastpath
\end{verbatim}
}
其中第一行是一个时间戳,第二行显示的是该监视器的版本信息,第三行和第四行分别是两种调用路径发生的情况,从显示的信息来看,第三行的调用路径最终在pipe\_wait这个函数中停留了1537个单位时间才返回,第四行的调用路径最终在do\_wait这个函数停留了65个单位时间之后返回。有了这样的信息,在发生问题的时候,我们能够更加方便地定位到问题发生的位置,这对性能测试和问题分析具有重大意义。


\subsubsection{系统监视器的实现}
前面已经介绍了在我们的框架中所使用的所有系统监视器,下面来介绍每一个系统监视器的具体实现方式。

前面提到在系统监视器的实现上,我们会遵循尽可能降低系统监视器对系统造成的影响的原则,来实现系统监视器。

为了方便后期的数据处理,对于每一个系统监视器,我们都要求每一秒的输出结果都必须附带一个时间戳作为不同的时间段数据的分隔,因此,对于每一个系统监视器,输出结构都要求保持下面的格式:
\begin{verbatim}
...
...
time: A
监视器在A时刻的输出结果
time: B
监视器在B时刻的输出结果
...
...
\end{verbatim}

于是,我们的测试框架中系统监视器的实现用伪代码可以表示为:






其中$Interval$这个变量就是两次输出之间的时间间隔,由于过高的采样速率会使系统监视器产生过高的额外性能损耗,于是我们使用的采样速率为每秒一次,因此在这里,$Interval$这个变量应该设置为1秒。

除了这些需要我们实现循环输出的系统监视器之外,还有一些Linux自带的工具,如iostat和vmstat等工具,都能够通过设置命令行参数来实现定时的循环输出效果,这样的系统监视器我们直接使用自带的循环输出功能,虽然时间戳的输出格式不同,但是只要在进行输出数据处理的过程中单独处理即可。

对于我们的测试框架中大部分的系统监视器来说,输出的信息就是/proc文件夹中的某一个虚拟文件,于是,我使用使用shell脚本进行实现,以此达到简单易懂的目的。

在实现的细节上,还需要注意的是,某一些系统监视器在使用之前必须要先将Linux操作系统中的相应功能开启,如latency\_stats,在读取/proc/latency\_stats之前还需要执行下面的命令开始内核中相应的功能:

\begin{lstlisting}[language=bash]
echo 0 > /proc/latency_stats
echo 1 > /proc/sys/kernel/latencytop
\end{lstlisting}

\subsection{系统监视器数据的格式化}

在上一个部分中,已经介绍了我们的测试框架中所使用的系统监视器,在本部分,将介绍系统监视器输出数据的提取方式。

虽然在实现系统监视器的时候,我们已经要求系统监视器按照一定的格式进行输出,但是这只能保证我们能够正确地读取到正确的时间戳和相应的系统监视器输出内容,不同的系统监视器输出由于所监视的目标不同和数据内容不同,相应地,输出数据的形式和格式也是互不相同的,如果直接对这样的数据进行处理和分析,将会增大分析和处理模块的负担,因此,在将数据进行进一步的处理和分析之前,我们需要将所有系统监视器的输出内容转换成方便识别和读取并且具有高度格式化的形式,这样,可以在一定程度上降低在处理和分析过程中的难度。

要进行系统监视器数据的格式化,主要有两个步骤:
\begin{enumerate}
\item 格式分析
\item 数据格式化
\end{enumerate}

下面,我们就详细描述系统监视器数据格式化的这两个过程。

\subsubsection{系统监视器输出格式分析}
分析系统监视器的输出格式,就是要了解系统监视器的输出数据中每一项数据所具体表示的含义和数据的结构,这样才能按照一定的规则将系统监视器的输出内容转换成指定的格式。

由于在我们框架中使用了较多的系统监视器,全部分析将占用大量的篇幅,因此,在本部分,我们仅挑选几个比较有代表性的系统监视器进行分析:

\begin{itemize}
\item 
\item 
\item 
\item 
\end{itemize}

\subsubsection{数据格式化}

\section{数据处理}

\subsection{数据分析与保存}


\subsection{数据的可视化比较}


