

\chapter{数据的提取与处理}

要检测性能回退是否出现,我们必须要从测试的环境中获取能够对性能进行判断的相关数据,即我们要进行测试数据的提取。

\section{数据提取}

虽然在已有的研究结果\ref{Jiang:2010:AAL:1831708.1831726}中指出使用额外的监视器来获取测试数据很有可能会造成额外的消耗从而影响测试数据的可靠性,但是在我们的测试框架中,我们采用最简单的方式来实现监视器,并且使用比较低的采样频率(一秒一次)来获取测试数据,以此来降低系统监视器对测试结果造成的影响。

\subsection{系统监视器}

系统监视器,就是以某种频率从系统中读取系统状态参数的并将其记录下来的程序。在这一部分,我将详细描述我们框架中的系统监视器的设计和实现。

前面提到,在我们的测试框架中,我们将采集尽可能多的,覆盖面广的测试数据,这样,我们才进行更加细致和全面的分析,于是在我们的测试框架中,将包含表\ref{tab:monitor_type}几类的系统监视器。

\begin{table}[tbp]
\centering  % 表居中
\tablehead{\hline
	监视器类型 & 监视器功能\\
	\hline \hline}
\tabletail{\hline}

\begin{supertabular}{p{3cm}p{10cm}}
CPU类 & 用于监视和CPU相关的系统状态\\
内存类 & 主要用于监视系统中和内存管理相关得系统参数\\
I/O类 & 主要用于监视系统中得输入输出状态\\
杂类 & 监控出了CPU,内存和I/O之外得其他参数\\
\end{supertabular}
\caption{监视器分类}
\label{tab:monitor_type}
\end{table}

下面我们将分类介绍这些不同的系统监视器。
\subsubsection{CPU类系统监视器}


对于CPU类的系统监视器来说,比较重要的自然是监视系统中的CPU的负载情况和CPU负载中的不同成分(包括内核系统负载和用户态负载),在多核SMP的情况下,我们还需要分别监视每一个核的运行情况,另外,在这里,我们把进程相关的监视器也包含在CPU类系统监视器中。

表\ref{tab:cpu_monitor}中列出了在我们的框架中所使用的所有CPU类系统监视器,以及它们所能够采集的数据。

\begin{table}[tbp]
\centering  % 表居中
\tablehead{\hline
	监视器 & 监视器功能 & 相关文件\\
	\hline \hline}
\tabletail{\hline}

\begin{supertabular}{p{3cm}p{7cm}p{3cm}}
interrupts & 监视系统中每一个CPU核中各种中断发生的情况 & /proc/interrupts\\
sched\_debug & 监视系统中每一个CPU核的各种调试参数和系统信息,如时钟的频率等;此外,该监视器还监视系统中的进程运行情况,提供进程调度的相关信息 & /proc/sched\_debug\\
softirqs & 监视系统中所有CPU核上的所有软中断情况 & /proc/softirqs\\
vmstat & 使用Linux中自带的常用工具vmstat来获取当前系统中的CPU负载状态,其中将包括 & 无\\
\end{supertabular}
\caption{CPU类系统监视器}
\label{tab:cpu_monitor}
\end{table}

在操作系统中,CPU作为执行指令的工具,其功能十分复杂,中断则是CPU最重要的功能之一,了解系统的中断有助于我们了解分析系统中相关硬件和相关子系统的运行状况,在发生性能回退问题的时候能够帮助我们进行更加准确地分析和定位问题。


\subsubsection{内存类系统监视器}

内存管理是操作系统中的一个重要功能,于是内存管理子系统也成为操作系统的中最为重要的子系统之一。内存管理子系统的性能好坏也直接影响整个操作系统的性能,因此,对于一个性能测试框架来说,内存管理性能相关的参数自然是必不可少的。

在Linux内核中,内存管理子系统是相当复杂的。其中,在内存管理中,比较重要的一个内存管理算法就是buddy算法,buddy system正是使用了这个这个算法来管理内存,这个算法的性能好坏也直接影响到整个内存管理系统,因此我们使用了一些监视器来专门监视buddy算法的运行状况。

此外,Linux内核的内存管理子系统还包括了非一致内存访问(NUMA)的管理。因此,除了监视普通的内存管理系统之外,还需要对NUMA管理系统也进行监视。

为了尽可能覆盖Linux内核中内存管理子系统中的相关功能,在我们的性能测试框架中,选择了表\ref{tab:mm_monitor}中列出的系统监视器为我们提供测试数据。

\begin{table}[tbp]
\centering  % 表居中
\tablehead{\hline
	监视器 & 监视器功能 & 相关文件\\
	\hline \hline}
\tabletail{\hline}

\begin{supertabular}{p{3cm}p{7cm}p{3cm}}
buddyinfo & 监视操作系统内存管理子系统中的Buddy内存分配算法的运行状况,其中的信息既可以用来进行性能的判断计算,还可以方便内核开发者的调试和问题定位 & /proc/buddyinfo\\
meminfo & 监视操作系统中内存各方面的使用状况 & /proc/meminfo\\
slabinfo & 监视内存管理子系统中SLAB算法的运行状况,功能和buddyinfo类似,可以提供测试数据,也可方便开发者的调试 & /proc/slabinfo\\
pagetypeinfo & 和buddyinfo监视器类似,其中还收集NUMA中不同node的详细信息 & /proc/pagetypeinfo\\
zoneinfo & 在NUMA系统中收集信息,可以得到NUMA中每一个node的详细信息 & /proc/zoneinfo\\
numa-meminfo & 监视NUMA系统中每一个node的内存信息 & /sys/devices/system/node\\
numa-vmstat & 收集NUMA系统中每一个node的虚拟内存信息 & /sys/devices/system/node\\
numa-numastat & 监视NUMA中每一个node的详细运行状态 & /sys/devices/system/node\\
proc-vmstat & 提供系统中相关内存子系统的相关信息,包括内存管理和页面切换管理等 &/proc/vmstat\\
\end{supertabular}
\caption{内存类系统监视器}
\label{tab:mem_monitor}
\end{table}



\subsubsection{I/O类系统监视器}

输入输出(I/O)几乎是计算机里最为重要的一部分了,同样,对于一个操作系统来说,输入输出(I/O)子系统同样也是最为重要的一部分。

一般,I/O子系统一般包括外设的输入输出和网络操作,我们选择的系统监视器也将监视这几个方面的测试数据。

表\ref{tab:io_monitor}中列出了我们的测试框架中使用的I/O类系统监视器。

\begin{table}[tbp]
\centering  % 表居中
\tablehead{\hline
	监视器 & 监视器功能 & 相关文件\\
	\hline \hline}
\tabletail{\hline}

\begin{supertabular}{p{3cm}p{7cm}p{3cm}}
mountstats & 监视操作系统中文件系统的挂载情况,其中会给出NFS文件系统的详细运行状态,这可以对操作系统中NFS的性能进行比较详细的监视 & /proc/self/mountstats\\
nfsstat & 这是Linux系统中常用的一款NFS状态查看工具,我们将其封装成一个完整的系统监视器,可以对NFS的运行状态进行详细的汇报,有助于进行性能分析 & 无\\
iostat & 同样也是一款Linux的常用工具,其作用在于 & 无\\
\end{supertabular}
\caption{I/O类系统监视器}
\label{tab:io_monitor}
\end{table}


\subsubsection{杂类系统监视器}

除了CPU类,内存类以及I/O类三类系统监视器之外,我们还选用了一些同样非常有用的监视器,这些监视器能够帮助我们更加方便和快速地分析和定位问题,表\ref{tab:misc_monitor}

\begin{table}[tbp]
\centering  % 表居中
\tablehead{\hline
	监视器 & 监视器功能 & 相关文件\\
	\hline \hline}
\tabletail{\hline}

\begin{supertabular}{p{3cm}p{7cm}p{3cm}}
lock\_stat &  & /proc/lock\_stat\\
latency\_stats & 同样也是一款Linux的常用工具,其作用在于 & /proc/latency\_stats\\
\end{supertabular}
\caption{杂类系统监视器}
\label{tab:misc_monitor}
\end{table}


\subsubsection{系统监视器的实现}

为了能够尽可能地降低系统监视器本身地额外消耗对测试结果带来地影响,我们将采用比较低的频率来进行系统参数的采样。


\subsection{系统监视器数据的提取}

\subsubsection{监视器输出格式分析}
\subsubsection{数据格式化}

\section{数据处理}

\subsection{数据分析与保存}


\subsection{数据的可视化比较}


